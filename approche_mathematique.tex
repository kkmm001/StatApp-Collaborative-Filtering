\documentclass[a4paper, 11pt]{article}
\usepackage[latin1]{inputenc}
\usepackage[T1]{fontenc}
\usepackage[francais]{babel}
\usepackage{amssymb,amsmath}
\allowdisplaybreaks
\usepackage{mathrsfs}
\usepackage{color}

\usepackage{graphicx}
%\usepackage{fancybox}
%\usepackage{cadre}
\newenvironment{bottompar}{\par\vspace*{\fill}}{\clearpage}

\usepackage[left=2.5cm,top=2.5cm,right=2.5cm,bottom=2.5cm,bindingoffset=0cm]{geometry}

\usepackage{fancyhdr}
\pagestyle{fancy}
\renewcommand{\footrulewidth}{0.5pt}
\fancyhead[R]{Groupe de statistique appliquée}
\fancyhead[L]{}
\fancyfoot[R]{ENSAE}
\fancyfoot[L]{2015-2016}

\newcommand{\thedate}{\today}
%\setlength\parindent{0em}

\begin{document}
\title{Sujet 2 - Collaborative filtering}
\author{
Etudiants : Biwei \textsc{Cui}, Claudia \textsc{Delgado}, Mehdi \textsc{Miah}, Ulrich \textsc{Mpeli Mpeli} \\
Tuteurs : Vincent \textsc{Cottet}, Mehdi  \textsc{Sebbar}
}

\date{\today}
\maketitle

\tableofcontents
\newpage

\begin{bottompar}
%\begin{flushleft}
%\includegraphics[scale=0.2]{logo_ENSAE.png}
Projet réalisé dans le cadre du Groupe de Statistique Appliquée
%\end{flushleft}
\end{bottompar}

\newpage

\section{Notations}

Nous considérons les bases de données suivantes : \\
$\diamond$ data.Ratings, la base contenant les notes des films ; \\
$\diamond$ data.Movies, la base décrivant les films ; \\
$\diamond$ data.Users, la base décrivant les utilisateurs. 

Nous noterons dans la suite $N_u$ le nombre d'utilisateurs, $N_m$ le nombre de films et $N_r$ le nombre de notes. \\
Par exemple, pour la base ml-100k : $N_u = 943$, $N_m = 1682$ et $N_r = 100000$.

\subsection{La base des notes : data.Ratings}

Montrons tout d'abord les premiers éléments de cette base pour ml-100k : 

% Table created by stargazer v.5.2 by Marek Hlavac, Harvard University. E-mail: hlavac at fas.harvard.edu
% Date and time: sam., déc. 12, 2015 - 17:15:36
\begin{table}[h!] \centering 
  \caption{data.Ratings} 
  \label{} 
\begin{tabular}{@{\extracolsep{2pt}} ccccc} 
\\[-1.8ex]\hline 
\hline \\[-1.8ex] 
 & userID & movieID & rating & timestamp \\ 
\hline \\[-1.8ex] 
1 & $196$ & $242$ & $3$ & $881,250,949$ \\ 
2 & $186$ & $302$ & $3$ & $891,717,742$ \\ 
3 & $22$ & $377$ & $1$ & $878,887,116$ \\ 
4 & $244$ & $51$ & $2$ & $880,606,923$ \\ 
5 & $166$ & $346$ & $1$ & $886,397,596$ \\ 
6 & $298$ & $474$ & $4$ & $884,182,806$ \\ 
7 & $115$ & $265$ & $2$ & $881,171,488$ \\ 
8 & $253$ & $465$ & $5$ & $891,628,467$ \\ 
9 & $305$ & $451$ & $3$ & $886,324,817$ \\ 
10 & $6$ & $86$ & $3$ & $883,603,013$ \\ 
\hline \\[-1.8ex] 
\end{tabular} 
\end{table}

Dans notre projet, nous ne considérerons pas la variable timestramp. \\

\subsection{La base des films : data.Movies}

Montrons tout d'abord les premiers éléments de cette base pour ml-100k : 

% Table created by stargazer v.5.2 by Marek Hlavac, Harvard University, and modified by the author of this file
\begin{table}[h!] \centering 
  \caption{data.Movies} 
  \label{} 
\begin{tabular}{@{\extracolsep{2pt}} cccc} 
\\[-1.8ex]\hline 
\hline \\[-1.8ex] 
 & movieID & title & date \\ 
\hline \\[-1.8ex] 
1 & $1$ & Toy Story (1995) & 01-Jan-1995 \\ 
2 & $2$ & GoldenEye (1995) & 01-Jan-1995 \\ 
3 & $3$ & Four Rooms (1995) & 01-Jan-1995 \\ 
4 & $4$ & Get Shorty (1995) & 01-Jan-1995 \\ 
5 & $5$ & Copycat (1995) & 01-Jan-1995 \\ 
\hline \\[-1.8ex] 
\end{tabular} 
\end{table}

Nous comptons également : \\
$\diamond$ la variable \texttt{IMDbURL} qui indique le lien url du film sur le site imdb.com ; \\
$\diamond$ 19 variables booléennes qui caractérisent le genre cinématographique : \texttt{unknown}, \texttt{action}, \texttt{adventure}, \texttt{animation}, \texttt{children's}, \texttt{comedy},
\texttt{crime}, \texttt{documentary}, \texttt{drama}, \texttt{fantasy}, \texttt{film-noir}, \texttt{horror}, \texttt{musical}, \texttt{mystery}, \texttt{romance}, \texttt{sci-fi}, \texttt{thriller}, \texttt{war}, \texttt{western}. \\
Ainsi, chaque film est caractérisé par 23 variables, dont 19 booléennes.

\subsection{La base des utilisateurs : data.Users}

Montrons tout d'abord les premiers éléments de cette base pour ml-100k : 

% Table created by stargazer v.5.2 by Marek Hlavac, Harvard University. E-mail: hlavac at fas.harvard.edu
% Date and time: sam., déc. 12, 2015 - 17:46:07
\begin{table}[h] \centering 
  \caption{data.Users} 
  \label{} 
\begin{tabular}{@{\extracolsep{5pt}} cccccc} 
\\[-1.8ex]\hline 
\hline \\[-1.8ex] 
 & userID & age & gender & occupation & zip.code \\ 
\hline \\[-1.8ex] 
1 & $1$ & $24$ & M & technician & 85711 \\ 
2 & $2$ & $53$ & F & other & 94043 \\ 
3 & $3$ & $23$ & M & writer & 32067 \\ 
4 & $4$ & $24$ & M & technician & 43537 \\ 
5 & $5$ & $33$ & F & other & 15213 \\ 
6 & $6$ & $42$ & M & executive & 98101 \\ 
7 & $7$ & $57$ & M & administrator & 91344 \\ 
8 & $8$ & $36$ & M & administrator & 05201 \\ 
9 & $9$ & $29$ & M & student & 01002 \\ 
10 & $10$ & $53$ & M & lawyer & 90703 \\ 
\hline \\[-1.8ex] 
\end{tabular} 
\end{table} 

\section{Prédiction}

Soit $U = (u_1, u_2, ..., u_{N_u})$ le vecteur des utilisateurs et $M = (m_1, m_2, ..., m_{N_m})$ le vecteur des films. \\
Considérons la matrice $Y = (y_{u,m})_{(u, m) \in U \times M}$ telle que :\\
$y_{u,m} = \text{note donnée par l'utilisateur } u \text{ pour le film } m \, ( y_{u,m} \in [\![1,5]\!])$.

$Y$ est donc une matrice $N_u \times N_m$. A quoi ressemble une telle matrice ?\\
Pour le cas du probleme ml-100k, $Y$ est une matrice de taille $943 \times 1 682$ comprenant 1 586 126 éléments dont exactement 100 000 valeurs non nulles ; donc le taux de complétion est de 6.3\%. C'est donc une matrice creuse.
\bigskip

Tentons de déterminer la note que donnerait l'utilisateur $u$ à un film $m$ qu'il n'a pas vu. \\
Introduisons pour cela les variables booléennes $r(u,m)_{(u, m) \in U \times M}$ telles que :\\
$r(u,m) = 
\left\{
\begin{array}{ll}
1 & \text{si l'utilisateur } u \text{ a noté le film } m \\
0  & \text{sinon} \
\end{array}
\right.$ 
\medskip 

Ainsi $y_{u,m}$ n'a de sens que si $r(u,m) =1$. Donc, tentons de déterminer les éléments de $Y=(y_{u,m)}$ tels que $r(u,m)=0$.

\subsection{Algorithmes naïves}

Considérons $u \in U, m \in M$ tels que $r(u,m)=0$.

\subsubsection{Aléatoire}

On affecte de manière aléatoire (suivant une distribution des notes ou une distribution uniforme sur $[0,5]$) une note à un élément $y_{u,m}$ vérifiant $r(u,m)=0$.

\subsubsection{Note unique : la moyenne de toutes les notes}

Cette seconde approche donne à tous les éléments de l'ensemble $\{y_{u,m} \vert r(u,m) =0 \}$ la valeur $\overline{y}$ définie par $\overline{y} \triangleq \displaystyle \dfrac{1}{N_r} \sum \limits_{u,m \vert r(u,m)=1} y_{u,m}$. \\
Dans le cas du problème ml-100k, $\overline{y} = ...$.

\subsubsection{Prédiction par la moyenne des moyennes des films}

\subsubsection{Prédiction par la moyenne des notes du film}

Soit $m_0 \in M$. On affecte dans ce cas la même note à l'ensemble $\{y_{u,m_0} \vert r(u,m_0)=0 \}$ la valeur $\overline{y_{(m_0)}}$ définie par $\overline{y_{(m_0)}} \triangleq  \displaystyle \dfrac{1}{\vert \{ r(u,m_0) = 1\} \vert } \sum \limits_{u \vert r(u,m_0)=1} y_{u,m_0}$. \\
$\overline{y_{(m_0)}}$ représente la note moyenne donnée au film $m_0$ par les autres utilisateurs. \\
$\vert \{ r(u,m_0) = 1\} \vert$ représente le nombre de notes qu'a reçues par le film $m_0$.
 
\subsubsection{Prédiction par la moyenne des notes de l'utilisateur}

De manière duale, soit $u_0 \in U$. On affecte ici la même note à l'ensemble $\{y_{u_0,m} \vert r(u_0,m)=0 \}$ la valeur $\overline{y_{(u_0)}}$ définie par $\overline{y_{(u_0)}} \triangleq \displaystyle \dfrac{1}{\vert \{ r(u_0,m) = 1\}\vert} \sum \limits_{m \vert r(u_0,m)=1} y_{u_0,m}$. \\
$\overline{y_{(u_0)}}$ représente la note moyenne donnée par l'utilisateur $u_0$. \\
$\vert \{ r(u_0,m) = 1\}\vert$ représente le nombre de notes qu'a émis l'utilisateur $u_0$.

\section{Quantification de l'erreur}

\subsection{L'erreur 0-1}

On considère que l'on a mal prédit une valeur dès que la prédiction est différente de la réalité.

\subsection{L'erreur RMSE}

\subsection{L'erreur MAD}

\section{Notion de similarité}

\subsection{La similarité de Jaccard}

\subsection{La similarité cosinus }

\end{document}
